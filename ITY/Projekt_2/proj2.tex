\documentclass[11pt,twocolumn]{article}

\usepackage[czech]{babel}
\usepackage[utf8]{inputenc}
\usepackage[IL2]{fontenc}
\usepackage[a4paper,left=1.5cm,text={18cm,25cm},top=2.5cm]{geometry}
\usepackage{times}
\usepackage{amsthm}
\usepackage{amsmath}
\usepackage{amssymb}

\theoremstyle{definition}
\newtheorem{define}{Definice}[section]
\newtheorem{algorithm}[define]{Algoritmus}
\theoremstyle{plain}
\newtheorem{sentence}{Věta}

\begin{document}
\begin{titlepage}
\begin{center}
\huge\textsc{Fakulta informačních technologií} \\
\smallskip 
\huge\textsc{vysokého učení technického v~Brně} \\
\vspace{\stretch{0.382}}
\LARGE{Typografie a~publikování -- 2.projekt} \\
\smallskip
\LARGE{Sazba dokumentů a~matematických výrazů} \\
\vspace{\stretch{0.618}}
\end{center}
{\LARGE 2017 \hfill Marek Kukučka}
\end{titlepage}


\section*{Úvod}
V~této úloze si vyzkoušíme sazbu titulní strany, matematických vzorců, prostředí a~dalších textových struktur obvyklých pro technicky zaměřené texty, například rovnice (\ref{equation1}) nebo definice \ref{num1} na straně \pageref{num1}.
\par
Na titulní straně je využito sázení nadpisu podle optického středu s~využitím zlatého řezu. Tento postup byl probírán na přednášce.
\section{Matematický text}
Nejprve se podíváme na sázení matematických symbolů a~výrazů v~plynulém textu. Pro množinu $V$ označuje card$(V)$ kardinalitu $V$. Pro množinu $V$ reprezentuje $V^*$ volný monoid generovaný množinou $V$ s~operací konkatenace. Prvek identity ve volném monoidu $V^*$ značíme symbolem $\varepsilon$. Nechť $V^+ =V^* - {\varepsilon}$. Algebraicky je tedy $V^+$ volná pologrupa generovaná množinou $V$ s~operací konkatenace. Konečnou neprázdnou množinu $V$ nazvěme  $abeceda$. Pro $w \in V^*$ označuje $|w|$ délku řetězce $w$. Pro $W \subseteq V$ označuje occur$(w, W)$ počet výskytů symbolů z~$W$ v~řetězci $w$ a~sym$(w, i)$ určuje $i$-tý symbol řetězce $w$; například sym$(abcd, 3) =c$.
\par
Nyní zkusíme sazbu definic a~vět s~využitím balíku \texttt{amsthm}.
\begin{define}\label{num1}
\emph {Bezkontextová gramatika} je čtveřice $G =(V,T,P,S)$, kde $V$ je totální abeceda, $T \subseteq V$ je abeceda terminálů, $S \in (V=T)$ je startující symbol a~$P$ je konečná množina \emph{pravidel} tvaru $q\colon A \rightarrow \alpha$, kde $A \in (V - T), \alpha \in V^*$ a~$q$ je návěští tohoto pravidla. Nechť $N = V - T$ značí abecedu neterminálu. Pokud $q\colon A \rightarrow \alpha \in P, \gamma, \delta \in V^*, G$ provádí derivační krok z~$\gamma A \delta$ do $\gamma\alpha\delta$ podle pravidla $q\colon A \rightarrow \alpha$, symbolicky píšeme $\gamma A \delta \Rightarrow \delta\alpha\gamma\ [q\colon A \rightarrow \alpha]$ nebo zjednodušeně $\gamma A \delta \Rightarrow \delta\alpha\gamma$. Standardním způsobem definujeme $\Rightarrow^m$, kde $m \geq 0$. Dále definujeme tranzitivní uzávěr $\Rightarrow^+$ a~tranzitivně-reflexivní uzávěr $\Rightarrow^*$.
\end{define}
\par
Algoritmus můžeme uvádět podobně jako definice textově, nebo využití pseudokódu vysázeného ve vhodném prostředí (například \texttt{algorithm2e}).
\begin{algorithm}
\emph{Algoritmus pro ověření bezkontextovosti gramatiky. Mějme gramatiku $G =(N,T,P,S)$}
\end{algorithm}
\begin{enumerate}
    \item \label{part1}
    \emph{Pro každé pravidlo $p \in P$ proveď test, zda $p$ na levé straně obsahuje právě jeden symbol z~$N$.}
    \item 
    \emph{Pokud všechna pravidla splňují podmínku z~kroku \ref{part1}, tak je gramatika $G$ bezkontextová.}
\end{enumerate}
\begin{define}
\emph{Jazyk} definovaný gramatikou $G$ definujeme jako $L(G) =\{w \in T^*|S \Rightarrow^* w\}$.
\end{define}
\subsection {Podsekce obsahující větu}
\begin{define}
Nechť $L$ je libovolný jazyk. $L$ je \emph{bezkontextový jazyk}, když a~jen když $L =L(G)$, kde $G$ je libovolná bezkontextová gramatika.
\end{define}
\begin{define}
Množině $\mathcal{L}_{CF} =\{L|L$ je bezkontextovy jazyk\} nazýváme třídou \emph{bezkontextových jazyků}.
\end{define}
\begin{sentence} \label{sentence1}
\emph{Nechť $L_{abc} =\{a^n b^n c^n |n \geq 0\}$. Platí, že $L_{abc} \notin \mathcal{L}_{CF}$.}
\end{sentence}
\begin{proof}
Důkaz se provede pomocí Pumping lemma pro bezkontextové jazyky, kdy ukážeme, že není možné, aby platilo, což bude implikovat pravdivost věty \ref{sentence1}.
\end{proof}
\section{Rovnice a~odkazy}

Složitější matematické formulace sázíme mimo plynulý text. Lze umístit několik výrazů na jeden řádek, ale pak je třeba tyto vhodně oddělit, například příkazem \verb|\quad|.

$$ \sqrt[x^2]{y^3_{0}} \quad \mathbb{N} =\{0,1,2,\ldots\} \quad x^{y^y} \neq x^{yy} \quad z_{i_j} \not\equiv z_{ij} $$

V~rovnici (\ref{equation1}) jsou využity tři typy závorek s~různou explicitně definovanou velkikostí.

\begin{align} \label{equation1}
 \begin{split}
     \bigg\{ \Big[ \big(a + b \big) * c \Big] ^d + 1 \bigg\}\quad=\quad x
 \end{split}
\end{align}
$$\lim_{ x \to \infty} \frac{\sin^2x + \cos^2x}{4}\quad=\quad y$$

V~této větě vidíme, jak vypadá implicitní vysázení limity $\lim_{n\to\infty} f(n)$ v~normálním odstavci textu. Podobně je to i~s~dalšími symboly jako $\sum_1^n$ či $\bigcup_{A \in \mathcal{B}}$. v~případě vzorce $\lim\limits_{x\to0}\frac{\sin x}{x}=1$ jsme si vynutili méně úspornou sazbu příkazem \verb|\limits|.

\begin{eqnarray}   
    \int\limits_{a}^{b} f\big(x\big)dx & = & -\int\limits_{a}^{b} f\big(x\big)dx \\
    \Big(\sqrt[5]{x^4}\Big)' = \Big(x^\frac{4}{5}\Big)' & = & \frac{4}{5}x^{-\frac{1}{5}} = \frac{4}{5\sqrt[5]{x}} \\
    \overline{\overline{A \lor B}} & = & \overline{\overline{A} \land \overline{B}}
\end{eqnarray}

\section {Matice}
Pro sázení matic se velmi často používá prostředí \verb|array| a~závorky (\verb|\left|,\verb|right|).

$$
\left(
\begin{array}{cc}
a + b & b - a \\
\widehat{\xi + \omega} & \hat\pi  \\
\vec{a} & \overset\longleftrightarrow{AC} \\
0 & \beta 
\end{array}
\right)
$$

$$
A = 
\left\| 
\begin{array}{cccc}
a_{11} & a_{12} & \ldots & a_{1n} \\ 
a_{21} & a_{22} & \ldots & a_{2n} \\
\vdots & \vdots & \ddots & \vdots \\
a_{m1} & a_{m2} & \ldots & a_{mn}
\end{array}
\right\|
$$

$$
\left|
\begin{array}{cc}
t & u \\
v & w \\
\end{array}
\right| 
= tw - uv
$$

Prostředí \verb|array| lze úspěsně využít i~jinde.

$$
\binom{n}{k}
=
\left\{
\begin{array}{ll}
\frac{n!}{k!(n-k)!} & \mbox{pro } 0 \leq k \leq n \\
0 & \mbox{pro } k < 0 \mbox{ nebo } k > n
\end{array}
\right.
$$

\section {Závěrem}

V~případě, že budete potřebovat vyjádřit matematickou konstrukci nebo symbol a~nebude se Vám dařit jej nalézt v~samotném \LaTeX u, doporučuji prostudovat možnosti balíku maker \AmS -\LaTeX. Analogická poučka platí obecně pro jakoukoli konstrukci v~\TeX u.
\end{document}