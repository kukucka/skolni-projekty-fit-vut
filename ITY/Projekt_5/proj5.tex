\documentclass{beamer}
 
\usepackage[utf8]{inputenc}
\usepackage[czech]{babel}
\usepackage[IL2]{fontenc}
\usepackage{times}
\usepackage{picture}
\usepackage{graphics}
\usepackage{bookman}
%\usepackage{pdflatex}
\usetheme{Madrid}
 
 
%Information to be included in the title page:
\title{Těsnopis}
\author{Marek Kukucka}
\institute{VUT v Brně}
\date{2017}

\begin{document}
\frame{\titlepage}
\begin{frame}
\frametitle{Co je to?}
\begin{block}{Definice}
Těsnopis je zjednodušené písmo určené k velmi rychlému, snadnějšímu a úspornějšímu zápisu informací.\end{block}
\end{frame}
\begin{frame}
\begin{itemize}
 \item Znaky využívané pro hlásky jsou oproti běžnému písmu značně zjednodušeny.
 \item Jako rozlišovací prvek často slouží poloha znaku vůči řádku a zesílení.
 \item Je užíván ke stručnému výtahu informací, ale i k doslovnému zápisu textu.
 \end{itemize}
\end{frame}

 \begin{frame}
 \frametitle{Ukázka}
 \begin{figure}
 \begin{center}
    \scalebox{0.9}{\includegraphics{abeceda.eps}}
 \end{center}
 \end{figure}
\end{frame}


 \begin{frame}
    \frametitle{Typy těsnopisných soustav}
    \begin{itemize}[<+->]
        \item \textbf{grafické}
         \begin{itemize}[<.->]
            \item kurzivní – odvozeny z latinské abecedy
            \item geometrické – odvozeny z geometrických symbolů
         \end{itemize}
        \item \textbf{strojové} 
          \begin{itemize}[<.->]
            \item  vznikly v moderní době, rozšířeny zejména v USA a západní Evropě
         \end{itemize}
    \end{itemize}   
 \end{frame}
 
 \begin{frame}
\frametitle{Český těsnopis}
   \begin{itemize}[<+->]
   \item Ve 40. letech 19. stoleti upravil pro češtinu Jakub Heger německou těsnopisnou soustavu Franze Xavera Gabelsbergera.
   \item Česká těsnopisná soustava sloužila často jako základ pro těsnopis ostatních slovanských národů.
   \item Vzhledem k rychlému rozvoji techniky je v současnosti těsnopis na ústupu.
   \item Snahy o vyvinutí strojové těsnopisné soustavu pro češtinu skončily neúspěchem.
   \end{itemize}
 \end{frame}
\begin{frame}
\frametitle{Zdroje}
 \begin{itemize}
     \item https://cs.wikipedia.org/wiki/T\%C4\%9Bsnopis
     \item http://www.featurepics.com/
 \end{itemize}
\end{frame}
\end{document}

